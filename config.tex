%\ifbooktrue
%\PassOptionsToClass{paper=a5}{scrbook}
%\fi
%

\PassOptionsToPackage{utf8}{inputenc}
\usepackage[utf8]{inputenc}
\usepackage[T1]{fontenc}
\usepackage[overload]{textcase}
\PassOptionsToPackage{french}{babel}
\usepackage[french]{babel}
\usepackage[autostyle,french=guillemets]{csquotes}
\MakeOuterQuote{"}
\usepackage{textcomp} % Fix warning with missing font shapes
\usepackage{scrhack} % Fix warnings when using KOMA with listings package  
\usepackage{xspace} % To get the spacing after macros right
\usepackage{mparhack} % To get marginpar right
\usepackage{fixltx2e} % Fixes some LaTeX stuff 
\usepackage{enumitem}
\PassOptionsToPackage{usenames,dvipsnames,svgnames,table}{xcolor}
\usepackage{epigraph}
\usepackage{setspace} % To change line in figures
\usepackage{amsmath}
\usepackage{mathtools}
\usepackage{lipsum}  
\usepackage[listings,dottedtoc,pdfspacing,eulermath,floatperchapter,beramono]{classicthesis} 
\usepackage{tocloft}
%----------------------------------------------------------------------
%	CHANGING TEXT AREA 
%----------------------------------------------------------------------
\usepackage[
	height=24cm,
	marginparwidth=2.5cm,
	outer=4.5cm,
	inner=1.5cm,
	top=2.9cm,
	bindingoffset=1.5cm,
	%showframe
]{geometry}
\setlength{\parskip}{.1\baselineskip}
\setlength{\cftfignumwidth}{4em} 
\setlength{\cfttabnumwidth}{4em} 
\setlength{\epigraphwidth}{.4\textwidth}


%------------------------------------------------------------
%	FLOATS: TABLES, FIGURES AND CAPTIONS SETUP
%------------------------------------------------------------
\usepackage{graphicx}
\usepackage{graphbox}
\usepackage{grffile}
\usepackage{transparent}

 

\graphicspath{{figures/introduction/},{figures/hiepaco/},{figures/corfish/},{figures/state/},{figures/kandinsky/}}

\usepackage{ifpdf}
\ifpdf
\DeclareGraphicsExtensions{.pdf,.png,.jpg,.jpeg}
\else
\DeclareGraphicsExtensions{.eps}
\fi
\usepackage{tabularx,booktabs,colortbl}
\usepackage{multirow,makecell}
\usepackage{rotating}
\setlength{\extrarowheight}{3pt} % Increase table row height
\newcommand{\tableheadline}[1]{\multicolumn{1}{c}{\spacedlowsmallcaps{#1}}}
\newcommand{\myfloatalign}{\centering} % To be used with each float for alignment
\usepackage{caption}
\usepackage{subcaption}
\usepackage{wrapfig}
\renewcommand{\floatpagefraction}{.8}% % Since the document will surely have large figures
\captionsetup{format=hang,font=small}
\captionsetup[subfigure]{justification=centering}
\captionsetup[subtable]{justification=centering}
\captionsetup[table]{format=hang}
\usepackage{sidenotes} % provides marginfigure, margintable, figure* (puts figure across page), sidenotetext and sidenote
\DeclareCaptionStyle{widefigure}{font=small}

% Extern copyright image
\newcommand{\archiveonly}[1]{
	\ifdiffusion\else #1 \fi
}


% Background image
\usepackage[pages=some]{background}

% Image with close-up
\newcommand{\withcloseup}[2]{
	\begin{tikzpicture}
		\node[inner sep=0pt,draw=gray,very thick] (full) at (0,0)
		{\includegraphics[width=#2]{#1}};
		\node[inner sep=0pt,draw=orange,very thick] (closeup) at (1,1)
		{
			\includegraphics[width=2.4cm,trim={1cm 1cm 20cm 20cm},clip]{#1}%
		};
		\draw[orange, thick] (-1.8,-2) rectangle (-.9, -1);
	\end{tikzpicture}
}
\usepackage[export]{adjustbox}

%------------------------------------------------------------
%	CODE & ALGO
%------------------------------------------------------------
\usepackage[boxed,french]{algorithm}
\usepackage[noend]{algpseudocode}
\usepackage{eqparbox} % For comment  alignment in algorithm
% New definitions
\algnewcommand\algorithmicswitch{\textbf{switch}}
\algnewcommand\algorithmiccase{\textbf{case}}
% New "environments"
\algdef{SE}[SWITCH]{Switch}{EndSwitch}[1]{\algorithmicswitch\ #1\ \algorithmicdo}{\algorithmicend\ \algorithmicswitch}%
\algdef{SE}[CASE]{Case}{EndCase}[1]{\algorithmiccase\ #1}{\algorithmicend\ \algorithmiccase}%
\algtext*{EndSwitch}%
\algtext*{EndCase}%
\algnewcommand{\LineComment}[1]{\State \textcolor{gray}{\(\triangleright\) #1}}
\algrenewcomment[1]{\hfill\eqparbox{COMMENT}{\textcolor{gray}{\(\triangleright\)  #1}}}

\renewcommand{\listalgorithmname}{Liste des algorithmes}
\floatname{algorithm}{Algorithme}
\renewcommand{\algorithmicreturn}{\textbf{retourne}}
\renewcommand{\algorithmicprocedure}{\textbf{procédure}}
\renewcommand{\algorithmicrequire}{\textbf{Entrée:}}
\renewcommand{\algorithmicensure}{\textbf{Sortie:}}
%\renewcommand{\algorithmiccomment}[1]{\{#1\}}
\renewcommand{\algorithmicend}{\textbf{fin}}
\renewcommand{\algorithmicif}{\textbf{si}}
\renewcommand{\algorithmicthen}{\textbf{alors}}
\renewcommand{\algorithmicelse}{\textbf{sinon}}
\renewcommand{\algorithmicfor}{\textbf{pour}}
\renewcommand{\algorithmicforall}{\textbf{pour tout}}
\renewcommand{\algorithmicdo}{\textbf{faire}}
\renewcommand{\algorithmicwhile}{\textbf{tant que}}
\renewcommand{\algorithmicreturn}{\textbf{retourner}}
\renewcommand{\algorithmicfunction}{\textbf{fonction}}
\newcommand{\algorithmicelsif}{\algorithmicelse\ \algorithmicif}
\newcommand{\algorithmicendif}{\algorithmicend\ \algorithmicif}
\newcommand{\algorithmicendfor}{\algorithmicend\ \algorithmicfor}



%------------------------------------------------------------
%	FONTES ET LANGUES
%------------------------------------------------------------
\usepackage{microtype}  
\usepackage[scale=0.9]{FiraMono}
\usepackage{Alegreya}
\renewcommand*\oldstylenums[1]{{\AlegreyaOsF #1}}
\usepackage{AlegreyaSans}
\usepackage[italic]{mathastext}
%------------------------------------------------------------
%	STYLE DES TITRES
%------------------------------------------------------------
\usepackage{titlesec}
\usepackage{titletoc}% http://ctan.org/pkg/titletoc
\newcommand{\hsp}{\hspace{20pt}}
\titleformat{\chapter}[hang]{\Huge\bfseries}{\sffamily\color{darkgray}\thechapter\hsp}{0pt}{\LARGE\bfseries\sffamily\scshape\textcolor{darkgray} }
\titleformat{\section}{\relax}{\Large\sffamily\bfseries\color{darkgray}\thesection}{.5em}{\sffamily\Large\bfseries\textcolor{darkgray}} 
\titleformat{\subsection}{\relax}{\large\sffamily\itshape\bfseries\color{darkgray}\thesubsection}{.5em}{\sffamily\large\itshape\bfseries\textcolor{darkgray}} 
\titleformat{\subsubsection}{\relax}{\large\sffamily\itshape\color{darkgray}\thesubsubsection}{.5em}{\sffamily\large\itshape\textcolor{darkgray}} 
\titleformat{\paragraph}[runin]{\normalfont\lsstyle\bfseries\scshape}{\theparagraph}{}{}
\renewcommand{\descriptionlabel}[1]{\hspace*{\labelsep}\bfseries\spacedlowsmallcaps{#1}}

% \usepackage{minitoc}
\titlecontents{chapter}
	[0pt] 
	{\addvspace{9pt}}
	{\scshape\lsstyle\thecontentslabel\quad}
	{\scshape\lsstyle\thecontentslabel\quad}
	{\hfill\contentspage}

%------------------------------------------------------------
%	DESSIN
%------------------------------------------------------------
\usepackage{tikz,tikz-3dplot}
\usepackage{etoolbox,xstring}
\usepackage{pgf,pgfplots,pgfplotstable,tikz-qtree}
\usepackage{overpic}
% Default style for PGFPlots
\pgfplotsset{
	compat=newest,
	grid=none,
	width=7cm,
	every axis plot/.append style={thick},
	cycle list name=exotic,
	legend style={
			at={(0.5,-0.2)},
			anchor=north,
			legend columns=2,
			cells={anchor=west},
			rounded corners=2pt,
		}
}
\usetikzlibrary{%
	3d,
	arrows,%
	arrows.meta,
	backgrounds,
	calc,%
	chains,%
	decorations.pathreplacing,%
	decorations.pathmorphing,% /pgf/decoration/random steps
	external,
	fit,
	graphs,
	graphs.standard,
	hobby,
	matrix,%
	patterns,
	positioning,% wg. " of "
	scopes,%
	shapes,
	shapes.misc,% wg. rounded rectangle
	shapes.arrows,%
	shapes.geometric,%
	shadows,%
	shadows.blur,
	trees,
	through
}
\usepackage[outline]{contour}
\contourlength{0.1em}
\pgfkeys{/pgf/number format/.cd,1000 sep={\,}, /pgf/number format/use comma}
\usepgfplotslibrary{units,groupplots,fillbetween}
% https://tex.stackexchange.com/questions/463814/check-whether-string-represents-valid-number-for-pgf
\newcommand{\isPGFNumber}[3]{% \isPGFNumber{<Number>}{<True>}{<False>}
	\begingroup
	\pgfkeys{/pgf/fpu/handlers/invalid number/.code={}}%
	\pgfmathfloatparsenumber{#1}%
	\global\pgfmathfloatgetflagstomacro\pgfmathresult\pgfFloatParseFlag%
	\endgroup
	\ifnum\pgfFloatParseFlag=3%
	#3%
	\else%
	#2%
	\fi%
}

% Layer management
\pgfdeclarelayer{foreground}
\pgfsetlayers{background,main,foreground}

\makeatletter
\tikzset{%
	on foreground layer/.style={%
		execute at begin scope={%
			\pgfonlayer{foreground}%
			\let\tikz@options=\pgfutil@empty%
			\tikzset{every on foreground layer/.try,#1}%
			\tikz@options%
		},
		execute at end scope={\endpgfonlayer}
	},
	on layer/.code={%
		\pgfonlayer{#1}\begingroup
		\aftergroup\endpgfonlayer
		\aftergroup\endgroup
	},
	node on layer/.code={%
		\gdef\node@@on@layer{%
			\setbox\tikz@tempbox=\hbox\bgroup\pgfonlayer{#1}\unhbox\tikz@tempbox\endpgfonlayer\egroup}
		\aftergroup\node@on@layer
	},
}
\def\node@on@layer{\aftergroup\node@@on@layer}


% Externalize tikz compilation
\usepackage{currfile-abspath}
\getmainfile

\usepackage{xpatch}

\makeatletter

\AtEndDocument{%
	\newwrite\maindeffile%
	\immediate\openout\maindeffile=\jobname.maindef%
	\immediate\write\maindeffile{%
		\noexpand\ifcsname mymainfile\noexpand\endcsname\noexpand\else%
		\noexpand\gdef\noexpand\mymainfile{\mymainfile}%
		\noexpand\fi%
	}%
	\immediate\closeout\maindeffile%
}

\@input{\jobname.maindef}

\ifcsname mymainfile\endcsname%
\ifx\mymainfile\@empty%
% Make a second attempt at extracting the real file name
% (necessary because currfile needs two runs on MiKTeX..)
\filename@parse{\themainfile}%
\let\mymainfile\filename@base%
\ifx\mymainfile\@empty%
% If still empty, fallback to default behavior
\def\mymainfile{\tikzexternal@realjob}%
\fi%
\fi%
\else%
% First attempt at extracting real file name
\filename@parse{\themainfile}%
\let\mymainfile\filename@base%
\fi%

\xpatchcmd{\tikzexternal@assemble@systemcall}%
{%
	\edef\texsource{%
		\string\def\string\tikzexternalrealjob{\tikzexternal@realjob}%
		\string\input{\tikzexternal@realjob}%
	}%
}%
{%
	\edef\texsource{%
		\string\def\string\tikzexternalrealjob{\tikzexternal@realjob}%
		\string\def\string\mymainfile{\mymainfile}%
		\string\input{\mymainfile}%
	}%
}%
{}
{}

\let\old@tikzexternalize@opt@withname\tikzexternalize@opt@withname%
\def\tikzexternalize@opt@withname[#1]#2{%
	% Suppress externalization as long as \mymainfile is empty
	% (This will only be the case the one or two first compilation runs!)
	\ifx\mymainfile\@empty\else%
	\old@tikzexternalize@opt@withname[#1]{#2}%
	\fi%
}

\makeatother % Fix for externalize + jobname -> https://tex.stackexchange.com/questions/273898/tikz-externalize-not-working-when-jobname-is-specified-for-main-document
\tikzexternalize[prefix=figures/cache/,up to date check=diff,only named]
\tikzsetfigurename{figure-}
%\usepackage[pspdf={-dDELAYSAFER}]{auto-pst-pdf}

%------------------------------------------------------------
%	SYMBOLES, MATHS, SURBRILLANCE
%------------------------------------------------------------
\usepackage[mathscr]{euscript}
\usepackage{fontawesome5,stmaryrd,mathbbol}
\usepackage{xfrac}


\usepackage{soul} % Text highlighting
\newcommand{\hlc}[2][yellow]{{%
			\colorlet{foo}{#1}%
			\sethlcolor{foo}\hl{#2}}%
}


%------------------------------------------------------------
%	FORMAT NUMBERS
%------------------------------------------------------------
\usepackage[round-mode=places,input-exponent-markers,binary-units]{siunitx}
\DeclareSIUnit{\octet}{o}
\sisetup{per-mode=symbol,per-symbol=\text{/},locale=FR}


%------------------------------------------------------------
%	CITATIONS REFERENCES
%------------------------------------------------------------
\usepackage[
	backend=biber,
	bibencoding=utf8,
	language=auto,%
	defernumbers=true, %
	% bibstyle=publist,
	% labelnumber=true,
	% plauthorhandling=highlight,
	% nameorder=given-family,
	% boldyear=false,
	maxbibnames=6, % default: 3, et al.
	maxcitenames=2,
	backref=true,%
	natbib=true % natbib compatibility mode (\citep and \citet still work)
]{biblatex}

% Clean up the bibtex rather than editing it
\AtEveryBibitem{
	\clearlist{address}
	\clearfield{date}
	\clearfield{eprint}
	%\clearfield{url}
	\clearlist{location}
	\clearfield{month}
	\clearfield{series}
	\clearfield{abstract}
	
	\ifentrytype{book}{}{% Remove publisher and editor except for books
		\clearlist{publisher}
		\clearname{editor}
	}
}

\defbibenvironment{bibnonum}
{\list
	{}
	{\setlength{\leftmargin}{\bibhang}%
		\setlength{\itemindent}{-\leftmargin}%
		\setlength{\itemsep}{\bibitemsep}%
		\setlength{\parsep}{\bibparsep}}}
{\endlist}
{\item}

% https://tex.stackexchange.com/questions/48400/biblatex-make-title-hyperlink-to-dois-url-or-isbn
% % https://tex.stackexchange.com/questions/48400/biblatex-make-title-hyperlink-to-dois-url-or-isbn
% % https://tex.stackexchange.com/questions/48400/biblatex-make-title-hyperlink-to-dois-url-or-isbn
% \input{clickable-title}

\newbibmacro{string+url}[1]{%
    \iffieldundef{url}{%
      \iffieldundef{doi}{%
        #1
      }{
        \href{https://doi.org/\thefield{doi}}{#1}%
      }%
    }{
      \href{\thefield{url}}{#1}%
    }%
}

\DeclareFieldFormat{title}{\usebibmacro{string+url}{\mkbibemph{#1}}}
\DeclareFieldFormat[article,inproceedings]{title}%
    {\usebibmacro{string+url}{\mkbibquote{#1}}}


\newbibmacro{string+url}[1]{%
    \iffieldundef{url}{%
      \iffieldundef{doi}{%
        #1
      }{
        \href{https://doi.org/\thefield{doi}}{#1}%
      }%
    }{
      \href{\thefield{url}}{#1}%
    }%
}

\DeclareFieldFormat{title}{\usebibmacro{string+url}{\mkbibemph{#1}}}
\DeclareFieldFormat[article,inproceedings]{title}%
    {\usebibmacro{string+url}{\mkbibquote{#1}}}


\newbibmacro{string+url}[1]{%
    \iffieldundef{url}{%
      \iffieldundef{doi}{%
        #1
      }{
        \href{https://doi.org/\thefield{doi}}{#1}%
      }%
    }{
      \href{\thefield{url}}{#1}%
    }%
}

\DeclareFieldFormat{title}{\usebibmacro{string+url}{\mkbibemph{#1}}}
\DeclareFieldFormat[article,inproceedings]{title}%
    {\usebibmacro{string+url}{\mkbibquote{#1}}}



% COLORS
\definecolor{Emerald}{RGB}{94,202,132}
\definecolor{Gray}{RGB}{127,127,127}
\definecolor{BurntOrange}{rgb}{0.8, 0.33, 0.0}
\definecolor{Froly}{RGB}{246, 114, 128}
\definecolor{WildWillow}{RGB}{182, 211,129}
\definecolor{nodeblue}{RGB}{0, 136, 170}
\definecolor{alizarin}{rgb}{0.82, 0.1, 0.26}
\definecolor{arsenic}{rgb}{0.23, 0.27, 0.29}
\definecolor{beaver}{rgb}{0.62, 0.51, 0.55}
\definecolor{asparagus}{RGB}{145,174,120}
\definecolor{brickred}{rgb}{0.8, 0.25, 0.33}
\definecolor{mygreen}{RGB}{177, 226, 213}
\definecolor{mypurple}{RGB}{190,164,252}
\definecolor{myyellow}{RGB}{251,239,191}
\definecolor{DeepBlue}{RGB}{0,102,255}
\definecolor{Greenish}{RGB}{37,138,126}
\definecolor{AppleGreen}{RGB}{172,213,5}
\definecolor{CandyPink}{RGB}{253,53,97}
\definecolor{turquoise}{RGB}{69,161,162}
\definecolor{yok}{RGB}{240,195,24}


%------------------------------------------------------------
% VARIABLES, COMMANDES 
%------------------------------------------------------------

\tikzset{>=triangle 45}

\newcommand{\myTitle}{Un long titre de thèse\xspace}
\newcommand{\EnglishTitle}{Same but in English\xspace}
\newcommand{\myName}{Prénom \textsc{Nom}\xspace}
\newcommand{\myProfs}{Francis \textsc{Marie} et Jackie \textsc{Paulette}\xspace}
\newcommand{\myUni}{LaBRI -- Université de Bordeaux\xspace}
\newcommand{\myTime}{\today\xspace}
\renewcommand{\myVersion}{version 0.9\xspace}
\author{\myName}
\title{\myTitle}

\newcommand{\ie}{c.-à-d.\,} %{\textit{i.\,e.\,}} % anglicisme
\newcommand{\eg}{p.\,ex.\,} %{\textit{e.\,g.\,}} % anglicisme
\newcommand{\cad}{c.-à-d.\,}
\newcommand{\cf}{cf.\ }

\newcounter{dummy} % Necessary for correct hyperlinks (to index, bib, etc.)
\newlength{\abcd} % for ab..z string length calculation

% Command to include externalized tikZ and SVG

\newcommand{\includetikz}[3][\linewidth]{
	\tikzsetnextfilename{#3}
	\resizebox{#1}{!}{
		\input{#2}
	}
}
\newcommand{\includesvg}[2][\linewidth]{
	\def\svgwidth{#1}
	\input{#2.pdf_tex}
}
	

\newcommand*\circled[1]{\tikzexternaldisable\tikz{
		\node[circle,font=\sffamily\scriptsize,draw=black,inner sep=0.5pt] {#1};}\tikzexternalenable}

% Redo underbrace command, broken with xetex apparently
\makeatletter
\def\underbrace#1{\@ifnextchar_{\tikz@@underbrace{#1}}{\tikz@@underbrace{#1}_{}}}
\def\tikz@@underbrace#1_#2{\tikz[baseline=(a.east)] {\node (a) {\(#1\)}; \draw[decorate,decoration={brace,amplitude=5pt}] (a.south east) -- node[below,inner sep=7pt] {\(\scriptstyle #2\)} (a.south west);}}
\makeatother


\newcommand{\checked}{\tikzexternaldisable\tikz \fill[asparagus] (0,0) rectangle (2ex,2ex);\tikzexternalenable} % 
\newcommand{\halfchecked}{\tikzexternaldisable\tikz \fill[asparagus] (0,0) -- (0,2ex) -- (2ex,2ex) --(0,0);\tikzexternalenable} % 
\newcommand{\crossed}{} % 
\newcommand{\bof}{\tikzexternaldisable\tikz \fill[gray!40] (0,0) rectangle (2ex,2ex);\tikzexternalenable}
\newcommand{\xtikz}[2][]{
	\tikzexternaldisable\tikz[#1,baseline=0]{#2}\tikzexternalenable
}
\newcommand{\smallbox}[1]{\xtikz{\fill[fill=#1] (0,0) rectangle (2ex,2ex);}}
\newcommand{\nope}{\xtikz{\draw[black] (0,0) -- (2ex,2ex);}}

\newcommand{\colorrow}[1]{\texttt{#1}&\smallbox{#1}}


%------------------------------------------------------------
% FIX PRETEX FOR EXTERNALIZED IMAGES
%------------------------------------------------------------
\ifcsname ifprint \endcsname%
\else%
\newif\ifprint\printfalse
\fi%

%------------------------------------------------------------
% HYPHENISATION
%------------------------------------------------------------
\usepackage[]{hyphenat}
\usepackage[shortcuts]{extdash} % Allow to hyphenate at "multi\-/dimensionnelle"
\hyphenation{ana-lyses con-tracté rem-plis-sage dé-formant con-duisent échel-le pré-ci-sément dia-gramme ma-téri-el-le-ment con-cerne meil-leurs cou-ples con-cep-tu-el-le nom-bre multi-forme agran-dis-se-ment exemp-ple prendre usu-el-le cha-que pren-dre for-mes for-me exem-ple}
\hbadness=99999  % for underfull badness (or any number >=10000)

%------------------------------------------------------------
% HYPERREFERENCES
%------------------------------------------------------------
% hyperref should be one of the last packages
% glossaries should come after hyperref
\PassOptionsToPackage{hyperfootnotes=false,pdfpagelabels}{hyperref}
\usepackage{hyperref}  % backref linktocpage pagebackref
\usepackage{url}
\ifprint

\hypersetup{
	colorlinks=false, linktocpage=false, pdfborder={0 0 0}, pdfstartpage=3, pdfstartview=FitV, 
	breaklinks=true, pdfpagemode=UseNone, pageanchor=true, pdfpagemode=UseOutlines,%
	plainpages=false, bookmarksnumbered, bookmarksopen=true, bookmarksopenlevel=1,%
	hypertexnames=true, pdfhighlight=/O,%nesting=true,
	frenchlinks,
	%------------------------------------------------
	% PDF file meta-information
	pdftitle={\myTitle},
	pdfauthor={\textcopyright\ \myName, \myUni},
	pdfsubject={},
	pdfkeywords={},
	pdfproducer={LaTeX}
	%------------------------------------------------
}

\else
\hypersetup{
	colorlinks=true, linktocpage=true, pdfstartpage=3, pdfstartview=FitV,
	breaklinks=true, pdfpagemode=UseNone, pageanchor=true, pdfpagemode=UseOutlines,%
	plainpages=false, bookmarksnumbered, bookmarksopen=true, bookmarksopenlevel=1,%
	hypertexnames=true, pdfhighlight=/O,%nesting=true,
	frenchlinks,
	urlcolor=teal, linkcolor=darkgray, citecolor=BurntOrange,
	%------------------------------------------------
	% PDF file meta-information
	pdftitle={\myTitle},
	pdfauthor={\textcopyright\ \myName, \myUni},
	pdfsubject={},
	pdfkeywords={},
	pdfproducer={LaTeX}
	%------------------------------------------------
}
\fi

\usepackage[french]{cleveref} 

%------------------------------------------------------------
% GLOSSAIRE 
%------------------------------------------------------------
\PassOptionsToPackage{smaller,printonlyused,withpage}{acronym} % Include printonlyused in the first bracket to only show acronyms used in the text
\usepackage[footnote,nogroupskip]{glossaries} % style=long,
\loadglsentries{front-back-matter/Glossary.tex}
\makenoidxglossaries



%% Page de garde
\newcommand{\ecoleDoctorale}{}
\newcommand{\specialite}{}
\newcommand{\annee}{}
\newcommand{\defensedate}{}
\newcommand{\jury}{}
\newcommand{\directors}{}

% École doctorale
\renewcommand{\ecoleDoctorale}{DE MATHÉMATIQUES ET D'INFORMATIQUE}

% Spécialité
\renewcommand{\specialite}{INFORMATIQUE}

% Disrecteur de thèse
\renewcommand{\directors}{\myProfs}

% Année de soutenance
\renewcommand{\annee}{1968}

% Date de la soutenance
\renewcommand{\defensedate}{23 juillet 1968}

\renewcommand{\jury}{
	% prénom, nom, titre, institution, rôle
	\exam{Juliette}{Smith}{Directeur de Recherche}
	{CNRS}{Rapportrice}
	\exam{Philippine}{Martin}{Professeur}
	{CNRS}{Rapportrice}
	\exam{Roger}{Dupont}{Professeur}
	{Université de Bordeaux}{Président du jury}
	\exam{Francis}{Marie}{Professeur}
	{Université de Bordeaux}{Co-Directeur}
	\exam{Jackie}{Paulette}{Maître de Conférences}
	{Université de Bordeaux}{Co-directeur}
	
}
%Titre: Maître de conférences, Professeur des universités, Chargé de recherche…
%Rôle: Président (du jury), Directeur, Co-directeur, Rapporteur, Examinateur…


