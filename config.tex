\usepackage[utf8]{inputenc}
\usepackage[T1]{fontenc}
\usepackage[overload]{textcase}
\usepackage[french]{babel}
\usepackage[autostyle,french=guillemets]{csquotes}
\MakeOuterQuote{"}
\usepackage{textcomp} % Fix warning with missing font shapes
\usepackage{scrhack} % Fix warnings when using KOMA with listings package  
\usepackage{xspace} % To get the spacing after macros right
\usepackage{mparhack} % To get marginpar right
\usepackage{fixltx2e} % Fixes some LaTeX stuff 
\usepackage{enumitem}
\usepackage[usenames,dvipsnames,svgnames,table]{xcolor}
\usepackage{epigraph}
\usepackage{setspace} % To change line in figures
\usepackage{amsmath}
\usepackage{mathtools}
\usepackage{lipsum}
\usepackage[listings,dottedtoc,pdfspacing,eulermath,floatperchapter,beramono]{classicthesis}
\usepackage{tocloft}

%----------------------------------------------------------------------
%	CHANGING TEXT AREA 
%----------------------------------------------------------------------
\usepackage[
	height=24cm,
	marginparwidth=2.5cm,
	outer=4.5cm,
	inner=1.5cm,
	top=2.9cm,
	bindingoffset=1.5cm,
	%showframe
]{geometry}
\setlength{\parskip}{.1\baselineskip}
\setlength{\cftfignumwidth}{4em}
\setlength{\cfttabnumwidth}{4em}
\setlength{\epigraphwidth}{.4\textwidth}


%------------------------------------------------------------
%	FLOATS: TABLES, FIGURES AND CAPTIONS SETUP
%------------------------------------------------------------
\usepackage{graphicx}
\usepackage{graphbox}
\usepackage{grffile}
\usepackage{transparent}
% Don't forget to update the list of figure folder if necessary
\graphicspath{{figures},{figures/chapter1/}, {figures/chapter2}, {figures/appendixA}}
\DeclareGraphicsExtensions{.pdf,.png,.jpg,.jpeg}
\usepackage{booktabs,tabularx,longtable,makecell}
\renewcommand\theadfont{\scshape}
\setlength{\extrarowheight}{3pt} % Increase table row height
\usepackage{caption,subcaption}
\renewcommand{\floatpagefraction}{.8} % Since the document will surely have large figures
\captionsetup{format=hang,font=small}
\captionsetup[subfigure]{justification=centering}
\captionsetup[subtable]{justification=centering}
\captionsetup[table]{format=hang}
\usepackage{sidenotes} % provides marginfigure, margintable, figure* (puts figure across page), sidenotetext and sidenote
\usepackage[export]{adjustbox}
\usepackage{csvsimple} % import csv file into tables

%------------------------------------------------------------
%	CODE & ALGO
%------------------------------------------------------------
\usepackage[boxed,french]{algorithm}
\usepackage[noend]{algpseudocode}
\usepackage{eqparbox} % For comment  alignment in algorithm
% New definitions
\algnewcommand\algorithmicswitch{\textbf{switch}}
\algnewcommand\algorithmiccase{\textbf{case}}
% New "environments"
\algdef{SE}[SWITCH]{Switch}{EndSwitch}[1]{\algorithmicswitch\ #1\ \algorithmicdo}{\algorithmicend\ \algorithmicswitch}%
\algdef{SE}[CASE]{Case}{EndCase}[1]{\algorithmiccase\ #1}{\algorithmicend\ \algorithmiccase}%
\algtext*{EndSwitch}%
\algtext*{EndCase}%
\algnewcommand{\LineComment}[1]{\State \textcolor{gray}{\(\triangleright\) #1}}
\algrenewcomment[1]{\hfill\eqparbox{COMMENT}{\textcolor{gray}{\(\triangleright\)  #1}}}
\renewcommand{\listalgorithmname}{Liste des algorithmes}
\floatname{algorithm}{Algorithme}
\renewcommand{\algorithmicreturn}{\textbf{retourne}}
\renewcommand{\algorithmicprocedure}{\textbf{procédure}}
\renewcommand{\algorithmicrequire}{\textbf{Entrée:}}
\renewcommand{\algorithmicensure}{\textbf{Sortie:}}
%\renewcommand{\algorithmiccomment}[1]{\{#1\}}
\renewcommand{\algorithmicend}{\textbf{fin}}
\renewcommand{\algorithmicif}{\textbf{si}}
\renewcommand{\algorithmicthen}{\textbf{alors}}
\renewcommand{\algorithmicelse}{\textbf{sinon}}
\renewcommand{\algorithmicfor}{\textbf{pour}}
\renewcommand{\algorithmicforall}{\textbf{pour tout}}
\renewcommand{\algorithmicdo}{\textbf{faire}}
\renewcommand{\algorithmicwhile}{\textbf{tant que}}
\renewcommand{\algorithmicreturn}{\textbf{retourner}}
\renewcommand{\algorithmicfunction}{\textbf{fonction}}
\newcommand{\algorithmicelsif}{\algorithmicelse\ \algorithmicif}
\newcommand{\algorithmicendif}{\algorithmicend\ \algorithmicif}
\newcommand{\algorithmicendfor}{\algorithmicend\ \algorithmicfor}

%------------------------------------------------------------
%	FONTES ET LANGUES
%------------------------------------------------------------
\usepackage{microtype}
\usepackage[scale=0.9]{FiraMono}
\usepackage{Alegreya}
\renewcommand*\oldstylenums[1]{{\AlegreyaOsF #1}}
\usepackage{AlegreyaSans}
\usepackage[italic]{mathastext}

%------------------------------------------------------------
%	STYLE DES TITRES
%------------------------------------------------------------
\usepackage{titlesec}
\usepackage{titletoc}% http://ctan.org/pkg/titletoc
\newcommand{\hsp}{\hspace{20pt}}
\titleformat{\chapter}[hang]{\Huge\bfseries}{\sffamily\color{darkgray}\thechapter\hsp}{0pt}{\LARGE\bfseries\sffamily\scshape\textcolor{darkgray} }
\titleformat{\section}{\relax}{\Large\sffamily\bfseries\color{darkgray}\thesection}{.5em}{\sffamily\Large\bfseries\textcolor{darkgray}}
\titleformat{\subsection}{\relax}{\large\sffamily\itshape\bfseries\color{darkgray}\thesubsection}{.5em}{\sffamily\large\itshape\bfseries\textcolor{darkgray}}
\titleformat{\subsubsection}{\relax}{\large\sffamily\itshape\color{darkgray}\thesubsubsection}{.5em}{\sffamily\large\itshape\textcolor{darkgray}}
\titleformat{\paragraph}[runin]{\normalfont\lsstyle\bfseries\scshape}{\theparagraph}{}{}
\renewcommand{\descriptionlabel}[1]{\hspace*{\labelsep}\bfseries\spacedlowsmallcaps{#1}}
\titlecontents{chapter}
[0pt]
{\addvspace{9pt}}
{\scshape\lsstyle\thecontentslabel\quad}
{\scshape\lsstyle\thecontentslabel\quad}
{\hfill\contentspage}
% \usepackage{minitoc}

%------------------------------------------------------------
%	DESSIN
%------------------------------------------------------------
\usepackage{tikz}
\usepackage{etoolbox,xstring}
\usepackage{pgf,pgfplots,pgfplotstable}
% Default style for PGFPlots
\pgfplotsset{
	compat=newest,
	grid=major,
	grid style={dashed,gray!30},
	width=7cm,
	every axis plot/.append style={thick},
	cycle list name=exotic,
	legend style={
			at={(0.5,1.2)},
			anchor=north,
			legend columns=2,
			cells={anchor=west},
			rounded corners=2pt,
		}
}
\usetikzlibrary{%
	arrows,%
	arrows.meta,
	backgrounds,
	calc,%
	external,
	fit,
	positioning,% wg. " of "
	shapes,
	shapes.misc,% wg. rounded rectangle
	through
}
\tikzset{>=triangle 45}
\pgfkeys{/pgf/number format/.cd,1000 sep={\,}, /pgf/number format/use comma}
\usepgfplotslibrary{units,groupplots,fillbetween}

% Externalize tikz compilation
\input{externalize-hack.tex} % Fix for externalize + jobname -> https://tex.stackexchange.com/questions/273898/tikz-externalize-not-working-when-jobname-is-specified-for-main-document
\tikzexternalize[prefix=figures/cache/,up to date check=diff,only named]
\tikzsetfigurename{figure-}
%\usepackage[pspdf={-dDELAYSAFER}]{auto-pst-pdf}

%------------------------------------------------------------
%	SYMBOLES, MATHS, SURBRILLANCE
%------------------------------------------------------------
\usepackage[mathscr]{euscript}
\usepackage{fontawesome5,stmaryrd,mathbbol}
\usepackage{xfrac}
\usepackage{soul} % Text highlighting
\newcommand{\hlc}[2][yellow]{{%
			\colorlet{foo}{#1}%
			\sethlcolor{foo}\hl{#2}}%
}


%------------------------------------------------------------
%	FORMAT NUMBERS
%------------------------------------------------------------
\usepackage[round-mode=places,input-exponent-markers,binary-units]{siunitx}
\DeclareSIUnit{\octet}{o}
\sisetup{per-mode=symbol,per-symbol=\text{/},locale=FR}

%------------------------------------------------------------
%	CITATIONS REFERENCES
%------------------------------------------------------------
\usepackage[
	backend=biber,
	bibencoding=utf8,
	language=auto,%
	defernumbers=true, %
	maxbibnames=6, % default: 3, et al.
	maxcitenames=2,
	backref=true,%
	natbib=true % natbib compatibility mode (\citep and \citet still work)
]{biblatex}

% Clean up the bibtex rather than editing it
\AtEveryBibitem{
	\clearlist{address}
	\clearfield{date}
	\clearfield{eprint}
	%\clearfield{url}
	\clearlist{location}
	\clearfield{month}
	\clearfield{series}
	\clearfield{abstract}

	\ifentrytype{book}{}{% Remove publisher and editor except for books
		\clearlist{publisher}
		\clearname{editor}
	}
}

\defbibenvironment{bibnonum}
{\list
	{}
	{\setlength{\leftmargin}{\bibhang}%
		\setlength{\itemindent}{-\leftmargin}%
		\setlength{\itemsep}{\bibitemsep}%
		\setlength{\parsep}{\bibparsep}}}
{\endlist}
{\item}

% https://tex.stackexchange.com/questions/48400/biblatex-make-title-hyperlink-to-dois-url-or-isbn
\newbibmacro{string+url}[1]{%
    \iffieldundef{url}{%
      \iffieldundef{doi}{%
        #1
      }{
        \href{https://doi.org/\thefield{doi}}{#1}%
      }%
    }{
      \href{\thefield{url}}{#1}%
    }%
}
\DeclareFieldFormat{title}{\usebibmacro{string+url}{\mkbibemph{#1}}}
\DeclareFieldFormat[article,inproceedings]{title}%
    {\usebibmacro{string+url}{\mkbibquote{#1}}}

\newboolean{bold}
\newcommand{\makeauthorsbold}[1]{%
  \DeclareNameFormat{author}{%
  \setboolean{bold}{false}%
    \renewcommand{\do}[1]{\expandafter\ifstrequal\expandafter{\namepartfamily}{####1}{\setboolean{bold}{true}}{}}%
    \docsvlist{#1}%
    \ifthenelse{\value{listcount}=1}
    {%
      {\expandafter\ifthenelse{\boolean{bold}}{\underline{\namepartgiven\addspace\namepartfamily}\addcomma\addspace}{\namepartgiven\addspace\namepartfamily\addcomma\addspace}}%
    }{\ifnumless{\value{listcount}}{\value{liststop}}
      {\expandafter\ifthenelse{\boolean{bold}}{\addcomma\addspace\underline{\namepartgiven\addspace\namepartfamily}}{\addcomma\addspace \namepartgiven\addspace\namepartfamily}}%
      {\expandafter\ifthenelse{\boolean{bold}}{\addcomma\addspace\underline{\namepartgiven\addspace\namepartfamily}\addspace\addcomma\isdot}{\addcomma\addspace\namepartgiven\addspace\namepartfamily\addcomma\isdot}}%
      }
    \ifthenelse{\value{listcount}<\value{liststop}}
    {\addcomma\space}{}
  }
}

%\makeauthorsbold{\myName} % Or replace with how your name is listed in your BiBTeX


% COLORS
% \definecolor{Emerald}{RGB}{94,202,132}
% \definecolor{Gray}{RGB}{127,127,127}
% \definecolor{BurntOrange}{rgb}{0.8, 0.33, 0.0}
% \definecolor{Froly}{RGB}{246, 114, 128}
% \definecolor{WildWillow}{RGB}{182, 211,129}
% \definecolor{nodeblue}{RGB}{0, 136, 170}
% \definecolor{alizarin}{rgb}{0.82, 0.1, 0.26}
% \definecolor{arsenic}{rgb}{0.23, 0.27, 0.29}
% \definecolor{beaver}{rgb}{0.62, 0.51, 0.55}
% \definecolor{asparagus}{RGB}{145,174,120}
% \definecolor{brickred}{rgb}{0.8, 0.25, 0.33}
% \definecolor{mygreen}{RGB}{177, 226, 213}
% \definecolor{mypurple}{RGB}{190,164,252}
% \definecolor{myyellow}{RGB}{251,239,191}
% \definecolor{DeepBlue}{RGB}{0,102,255}
% \definecolor{Greenish}{RGB}{37,138,126}
% \definecolor{AppleGreen}{RGB}{172,213,5}
% \definecolor{CandyPink}{RGB}{253,53,97}
% \definecolor{turquoise}{RGB}{69,161,162}
% \definecolor{yok}{RGB}{240,195,24}


%------------------------------------------------------------
% VARIABLES, COMMANDES 
%------------------------------------------------------------
\newcommand{\myTitle}{Un long titre de thèse\xspace}
\newcommand{\EnglishTitle}{Same but in English\xspace}
\newcommand{\myName}{Prénom \textsc{Nom}\xspace}
\newcommand{\myProfs}{Francis \textsc{Marie} et Jackie \textsc{Paulette}\xspace}
\newcommand{\myUni}{LaBRI -- Université de Bordeaux\xspace}
\newcommand{\myTime}{\today\xspace}
\renewcommand{\myVersion}{version 0.9\xspace}
\author{\myName}
\title{\myTitle}

\newcommand{\ie}{c.-à-d.\,} %{\textit{i.\,e.\,}} % anglicisme
\newcommand{\eg}{p.\,ex.\,} %{\textit{e.\,g.\,}} % anglicisme
\newcommand{\cad}{c.-à-d.\,}
\newcommand{\cf}{cf.\ }

\newcounter{dummy} % Necessary for correct hyperlinks (to index, bib, etc.)
\newlength{\abcd} % for ab..z string length calculation

% Command to include externalized tikZ and SVG
\newcommand{\includetikz}[3][\linewidth]{
	\tikzsetnextfilename{#3}
	\resizebox{#1}{!}{
		\input{#2}
	}
}
\newcommand{\includesvg}[2][\linewidth]{
	\def\svgwidth{#1}
	\input{#2.pdf_tex}
}

% Redo underbrace command, broken with xetex apparently
\makeatletter
\def\underbrace#1{\@ifnextchar_{\tikz@@underbrace{#1}}{\tikz@@underbrace{#1}_{}}}
\def\tikz@@underbrace#1_#2{\tikz[baseline=(a.east)] {\node (a) {\(#1\)}; \draw[decorate,decoration={brace,amplitude=5pt}] (a.south east) -- node[below,inner sep=7pt] {\(\scriptstyle #2\)} (a.south west);}}
\makeatother

\newcommand{\checked}{\tikzexternaldisable\tikz \fill[asparagus] (0,0) rectangle (2ex,2ex);\tikzexternalenable} % 
\newcommand{\halfchecked}{\tikzexternaldisable\tikz \fill[asparagus] (0,0) -- (0,2ex) -- (2ex,2ex) --(0,0);\tikzexternalenable} % 
\newcommand{\crossed}{} % 
\newcommand{\bof}{\tikzexternaldisable\tikz \fill[gray!40] (0,0) rectangle (2ex,2ex);\tikzexternalenable}
\newcommand{\xtikz}[2][]{
	\tikzexternaldisable\tikz[#1,baseline=0]{#2}\tikzexternalenable
}
\newcommand{\smallbox}[1]{\xtikz{\fill[fill=#1] (0,0) rectangle (2ex,2ex);}}
\newcommand{\nope}{\xtikz{\draw[black] (0,0) -- (2ex,2ex);}}
\newcommand{\colorrow}[1]{\texttt{#1}&\smallbox{#1}}


%------------------------------------------------------------
% FIX PRETEX FOR EXTERNALIZED IMAGES
%------------------------------------------------------------
\ifcsname ifprint \endcsname%
\else%
\newif\ifprint\printfalse
\fi%

%------------------------------------------------------------
% HYPHENISATION
%------------------------------------------------------------
\usepackage[]{hyphenat}
\usepackage[shortcuts]{extdash} % Allow to hyphenate at "multi\-/dimensionnelle"
\hyphenation{ana-lyses con-tracté rem-plis-sage dé-formant con-duisent échel-le pré-ci-sément dia-gramme ma-téri-el-le-ment con-cerne meil-leurs cou-ples con-cep-tu-el-le nom-bre multi-forme agran-dis-se-ment exemp-ple prendre usu-el-le cha-que pren-dre for-mes for-me exem-ple}
\hbadness=99999  % for underfull badness (or any number >=10000)

%------------------------------------------------------------
% HYPERREFERENCES
%------------------------------------------------------------
% hyperref should be one of the last packages
% glossaries should come after hyperref
\PassOptionsToPackage{hyperfootnotes=false,pdfpagelabels}{hyperref}
\usepackage{hyperref}  % backref linktocpage pagebackref
\usepackage{url}
\ifprint

	\hypersetup{
		colorlinks=false, linktocpage=false, pdfborder={0 0 0}, pdfstartpage=3, pdfstartview=FitV,
		breaklinks=true, pdfpagemode=UseNone, pageanchor=true, pdfpagemode=UseOutlines,%
		plainpages=false, bookmarksnumbered, bookmarksopen=true, bookmarksopenlevel=1,%
		hypertexnames=true, pdfhighlight=/O,%nesting=true,
		frenchlinks,
		%------------------------------------------------
		% PDF file meta-information
		pdftitle={\myTitle},
		pdfauthor={\textcopyright\ \myName, \myUni},
		pdfsubject={},
		pdfkeywords={},
		pdfproducer={LaTeX}
		%------------------------------------------------
	}

\else
	\hypersetup{
		colorlinks=true, linktocpage=true, pdfstartpage=3, pdfstartview=FitV,
		breaklinks=true, pdfpagemode=UseNone, pageanchor=true, pdfpagemode=UseOutlines,%
		plainpages=false, bookmarksnumbered, bookmarksopen=true, bookmarksopenlevel=1,%
		hypertexnames=true, pdfhighlight=/O,%nesting=true,
		frenchlinks,
		urlcolor=teal, linkcolor=darkgray, citecolor=BurntOrange,
		%------------------------------------------------
		% PDF file meta-information
		pdftitle={\myTitle},
		pdfauthor={\textcopyright\ \myName, \myUni},
		pdfsubject={},
		pdfkeywords={},
		pdfproducer={LaTeX}
		%------------------------------------------------
	}
\fi

\usepackage[french]{cleveref}

%------------------------------------------------------------
% GLOSSAIRE 
%------------------------------------------------------------
\PassOptionsToPackage{smaller,printonlyused,withpage}{acronym} % Include printonlyused in the first bracket to only show acronyms used in the text
\usepackage[footnote,nogroupskip]{glossaries} % style=long,
\loadglsentries{front-back-matter/Glossary.tex}
\makenoidxglossaries


%------------------------------------------------------------
% PAGE DE GARDE 
%------------------------------------------------------------
\newcommand{\ecoleDoctorale}{DE MATHÉMATIQUES ET D'INFORMATIQUE}
\newcommand{\specialite}{INFORMATIQUE}
\newcommand{\directors}{\myProfs}
\newcommand{\annee}{1968}
\newcommand{\defensedate}{23 juillet 1968}
\newcommand{\jury}{
	% prénom, nom, titre, institution, rôle
	\exam{Juliette}{Smith}{Directeur de Recherche}
	{CNRS}{Rapportrice}
	\exam{Philippine}{Martin}{Professeur}
	{CNRS}{Rapportrice}
	\exam{Roger}{Dupont}{Professeur}
	{Université de Bordeaux}{Président du jury}
	\exam{Francis}{Marie}{Professeur}
	{Université de Bordeaux}{Co-Directeur}
	\exam{Jackie}{Paulette}{Maître de Conférences}
	{Université de Bordeaux}{Co-directeur}

}
%Titre: Maître de conférences, Professeur des universités, Chargé(e) de recherche…
%Rôle: Président(e) (du jury), Directeur/Directrice, Rapporteur/Rapportrice, Examinateur/Examinatrice


