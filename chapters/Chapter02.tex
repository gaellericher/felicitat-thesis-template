
\chapter{Chapitre 2}
\label{ch-2}

\epigraph{C'est la mobilité interne de l'image qui caractérise la graphie moderne. On ne "dessine" plus un graphique une fois pour toutes. On le "construit" et on le reconstruit (on le manipule) jusqu'au moment où toutes les relations qu'il recèle ont été perçues.}{---Jacques Bertin, \textit{La graphique et le traitement graphique de l'information}}


\begin{figure}
	\centering
	% Horizon penetrating coordinates (vs. Schwarzschild coordinates)
% for a black hole spacetime, with excision
% Author: Jonah Miller
% https://texample.net/tikz/examples/spacetime/
\tikzset{zigzag/.style={decorate, decoration=zigzag}}
\def \L {2.}

\begin{tikzpicture}
  % causal diamond
  \draw[thick,red,zigzag] (-\L,\L) coordinate(stl) -- (\L,\L) coordinate (str);
  \draw[thick,black] (\L,-\L) coordinate (sbr)
    -- (0,0) coordinate (bif) -- (stl);
  \draw[thick,black,fill=blue, fill opacity=0.2,text opacity=1] 
    (bif) -- (str) -- (2*\L,0) node[right] (io) {$i^0$} -- (sbr);

  % null labels
  \draw[black] (1.4*\L,0.7*\L) node[right]  (scrip) {$\mathcal{I}^+$}
               (1.5*\L,-0.6*\L) node[right] (scrip) {$\mathcal{I}^-$}
               (0.2*\L,-0.6*\L) node[right] (scrip) {$\mathcal{H}^-$}
               (0.5*\L,0.85*\L) node[right] (scrip) {$\mathcal{H}^+$};

  % singularity label
  \draw[thick,red,<-] (0,1.05*\L) 
    -- (0,1.2*\L) node[above] {\color{red} singularity};
  % Scwharzschild surface
  \draw[thick,blue] (bif) .. controls (1.*\L,-0.35*\L) .. (2*\L,0);
  \draw[thick,blue,<-] (1.75*\L,-0.1*\L)  -- (1.9*\L,-0.5*\L)
    -- (2*\L,-0.5*\L) node[right,align=left]
    {$t=$ constant\\in Schwarzschild\\coordinates};
  % excision surface
  \draw[thick,dashed,red] (-0.3*\L,0.3*\L) -- (0.4*\L,\L);
  \draw[thick,red,<-] (-0.33*\L,0.3*\L) 
    -- (-0.5*\L,0.26*\L) node[left,align=right] {excision\\surface};
  % Kerr-Schild surface
  \draw[darkgreen,thick] (0.325*\L,0.325*\L) .. controls (\L,0) .. (2*\L,0);
  \draw[darkgreen,dashed,thick] (0.325*\L,0.325*\L) -- (-0.051*\L,0.5*\L);
  % Kerr-Schild label
  \draw[darkgreen,thick,<-] (0.95*\L,0.15*\L) -- (1.2*\L,0.5*\L)
    -- (2*\L,0.5*\L) node[right,align=left]
    {$\tau=$ constant\\in Kerr-Schild\\coordinates};
\end{tikzpicture}

	\caption[Singularité.]{Quelque chose à propos des fameux \glsplural{trou_noir}.}
	\label{fig-trou}
\end{figure}

La \cref{fig-trou} est créé avec TikZ à la différence de la figure \ref{fig-bertin-matrice} (\cpageref{fig-bertin-matrice}).


\lipsum[2-4]

