
\chapter{Chapitre 2}
\label{ch-2}

\epigraph{C'est la mobilité interne de l'image qui caractérise la graphie moderne. On ne "dessine" plus un graphique une fois pour toutes. On le "construit" et on le reconstruit (on le manipule) jusqu'au moment où toutes les relations qu'il recèle ont été perçues.}{---Jacques Bertin, \textit{La graphique et le traitement graphique de l'information}}


\begin{figure}
	\centering
	\input{figures/chapter2/black_hole}
	\caption[Singularité.]{Quelque chose à propos des fameux \glsplural{trou_noir}.}
	\label{fig-trou}
\end{figure}

La \cref{fig-trou} est créé avec TikZ à la différence de la figure \ref{fig-bertin-matrice} (\cpageref{fig-bertin-matrice}).


\lipsum[2-4]

