
%- roll-up/ drille-down
%- cécité au changement
%- focus+context
%- fisheye
%- vues multiples coordonnées  
%- highlighting / surbrillance
%- vue d'ensemble
%- zoom
%- vue
%- encodage visuel
%- idiome
%- parallélisme
%- marque
%- variable visuelle
%- utilisateur
%- interaction
%- tâche
%- complexité
%- latence et bande passante
%- latence d'interaction
%- réactivité
%- réduction dimensionnelle
%- problème de l'immobilier de l'espace écran
%- affordance

\newglossaryentry{surbrillance}
{
	name=surbrillance,
	description={(\emph{highlighting}) mise en avant d'entités par altération d'une variable visuelle}
}

\newglossaryentry{partitionnement}
{
        name=partitionnement,
        description={(\emph{clustering}) partition de données calculée automatique en utilisant une notion de similarité entre éléments}
}


\newglossaryentry{brossage_lien}
{
        name=brossage et lien,
        description={(\emph{brushing \& linking}, aussi \emph{linked highlighting}) technique d'interaction permettant à l'utilisateur de définir une sélection sur une partie de la visualisation provoquant la mise en surbrillance de toutes les instances liées à cette sélection}
}
 
\newglossaryentry{jittering}
{
        name=\emph{jittering},
        description={déplacement des entités visuelles autour de leur position d'origine ayant pour objectif d'éviter leur superposition et d'augmenter leur visibilité}
}

\newglossaryentry{treemap}
{
	name=\emph{treemap},
	description={représentation visuelle de données hiérarchique (arbre) utilisant la propriété d'inclusion géomérique (entre des rectangles ou cercles généralement) pour représenter le lien de parenté entre deux entités }
}


\newglossaryentry{encombrement_visuel}
{
        name=encombrement visuel,
        description={(\emph{visual clutter}) accumulation excessive locale ou globale  d'éléments graphiques dont la densité voire la superposition peut nuire à la compréhension de la visualisation  voire dissimuler des éléments jusqu'à entraîner une perte d'information  }
}

\newglossaryentry{binning}
{
        name=\emph{binning},
        description={partitionnement par pavage régulier de l'espace (avec des rectangles ou hexagones)  qui sert en général à partitionner des données pour les représenter de manière agrégée }
}

\newglossaryentry{marque}
{
	name=marque,
	plural={marques},
	description={élément visuel (ou graphique)}
}

\newglossaryentry{gamut}
{
        name=gamut,
        description={ensemble des couleurs qu'un dispositif (écran) permet de reproduire}
}

 \newglossaryentry{teinte}
{
	name=teinte,
	description={(\emph{hue}) couleur perçue, aussi appelé \emph{ton}. Par exemple: "orange", "rouge", etc}
}



\newglossaryentry{kde}
{
        name=KDE,
%	    first={estimation par noyaux de la densité (\emph{kernel density estimation}, \glsentrytext{kde})},
        description={estimation par noyaux de la densité (\emph{kernel density estimation}). Fournit une fonction estimant la densité de probabilité d'une variable aléatoire en tout point d'un domaine, à partir d'un échantillon. L'estimation dépend du choix du \emph{noyau} et de la \emph{fenêtre}, paramètre régissant le degré de lissage de l'estimation. Le noyau est souvent une fonction gaussienne standard  }
}


\newacronym{ndd}{NDD}{niveau de détail}
\newacronym{doi}{DOI}{degré d'intérêt (\emph{degree of interest})}
\newacronym{cpu}{CPU}{processeur (\emph{central processing unit})}
\newacronym{gpu}{GPU}{processeur graphique (\emph{graphics processing unit})}
\newacronym{cdd}{CDD}{carte de densité discrète}
\newacronym{ssim}{SSIM}{Structural SIMilarity}
